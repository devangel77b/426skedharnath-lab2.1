\section{Discussion}

\textbf{Interpretation of Results:}

\paragraph{Proportional Relationship and Validation of Newton's Second Law:}

The graph demonstrates a clear linear relationship between the mass ratio \( \frac{m_2}{m_1 + m_2} \) and the acceleration \( a \). As the mass ratio increases, the acceleration increases proportionally, consistent with the theoretical model \( a = \frac{m_2 g}{m_1 + m_2} \). The data points closely follow the expected trend, and the slope of the best-fit line is consistent with the value of the gravitational constant \( g \). Small deviations from the line may be attributed to measurement errors, such as timing inaccuracies or frictional losses in the pulley system. 

While there are minor discrepancies between the calculated and predicted values, the results are generally in agreement, validating the proportionality between the applied force and acceleration as predicted by Newton's Second Law. The linear relationship observed in the graph confirms that the experiment supports the theoretical model.

For a more precise comparison, linear regression analysis could be performed on the data to quantify the goodness of fit and provide a more detailed measure of the correlation between force and acceleration.
 

\section{Sources of Error and Implications:}

\paragraph{Friction and Air Resistance:}

The actual acceleration observed in the experiment may have been slightly lower than predicted due to the effects of friction between the cart and the track, as well as air resistance acting on the system. These factors reduce the net force acting on the cart and could explain any discrepancies between the observed and predicted accelerations.

\paragraph{Timing Accuracy:}

The accuracy of the recorded time for each trial is affected by human reaction time when using a stopwatch. This introduces a small amount of error that could influence the precision of the calculated acceleration. To reduce this error, automated timing devices or motion sensors could be employed in future experiments, ensuring more consistent and accurate timing measurements.

\paragraph{Pulley Efficiency:}

The pulley system itself may introduce some resistance due to friction or mechanical inefficiencies, which would slightly reduce the net force experienced by the cart. This could contribute to a smaller acceleration than theoretically expected, and future experiments could attempt to minimize these losses by using a more efficient pulley system.

\section{Conclusion:}

While the data generally supports the hypothesis of a proportional relationship between force and acceleration, a more thorough analysis, including statistical tools like regression analysis and error propagation, would provide a clearer understanding of the relationship and any deviations from the expected results.

This experiment successfully illustrates the direct relationship between net force and acceleration in a pulley system, affirming Newton's Second Law of Motion [1] in a practical setup. The data clearly demonstrate that as net force on the cart increases, so does its acceleration, supporting our hypothesis and highlighting the predictable nature of classical mechanics.

For future extensions, teams could investigate the effects of additional variables, such as frictional forces by using various surfaces under the cart, or examine the impact of pulley efficiency by experimenting with different pulley materials and designs. These modifications would offer a more comprehensive understanding of real-world factors that influence force and acceleration relationships in mechanical systems.
