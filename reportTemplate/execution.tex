\section{Methods and Materials}

\subsection{Materials}
The materials used in this experiment were as follows:

1. A minimal friction cart from PASCO with  an additional mass on top for a total mass of \( m_1 = 780 \, \text{g} \)

2. Multiple hanging masses \( m_2 = 20 \, \text{g}, 50 \, \text{g}, 100 \, \text{g}, 200 \, \text{g}, 500 \, \text{g} \)

3. A pulley system

4. A string connecting the cart and hanging mass

5. A stopwatch for measuring time

6. A measuring tape for measuring distance

7. A scale for measuring mass

8. A minimal friction track with PASCO-labeled measurements

\begin{figure}[H]
    \centering
    \includegraphics[width=1\linewidth]{Untitled drawing (1).png}
    \caption{Free body diagram}
    \label{fig:freebody}
\end{figure}

\begin{figure}[H]
    \centering
    \includegraphics[width=0.7\linewidth]{physics_setup.png}
    \caption{Physical Setup}
    \label{fig:setup}
\end{figure}




\subsection{Procedure}

The experiment was conducted as follows:

\begin{enumerate}
    \item A low-friction track was set up with a pulley system attached at one end. A cart (\(m_1\)) was placed on the track and connected to a hanging mass (\(m_2\)) using a string that passed over the pulley. This arrangement can be seen above in Figure 1, as it is a modified Atwood pulley system.
    \item The hanging mass was suspended vertically off the edge of the track, ensuring that the string remained taut and aligned with the pulley to minimize friction.
    \item The hanging mass was released, allowing the cart to accelerate along the track due to the force exerted by \(m_2\).
    \item The time (\(t\)) taken for the cart to travel a specified distance (\(d\)) was measured using stopwatches and video recordings. The distance \(d\) was predetermined and marked on the track.
    \item The acceleration (\(a\)) of the cart was calculated using the kinematic equation:
    \[
    a = \frac{2d}{t^2}
    \]
    where \(d\) represents the distance traveled, and \(t\) represents the measured time.
    \item The procedure was repeated for multiple configurations of \(m_2\) by varying its mass. For each configuration, the acceleration was recorded and averaged over multiple trials to improve accuracy.
\end{enumerate}

Further details regarding the derivation of the acceleration formula can be found in the Appendix.

