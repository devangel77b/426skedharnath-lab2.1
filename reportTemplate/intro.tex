\section{Introduction}
The fundamental principles of classical mechanics provide a framework for understanding the motion of objects under the influence of various forces. One of the most crucial relationships in this field is encapsulated in Newton's Second Law of Motion[1], which states that the acceleration of an object is directly proportional to the net force acting on it and inversely proportional to its mass. This relationship can be mathematically expressed as:

\begin{equation}
    F = ma
\end{equation}

where \( F \) is the net force applied to an object, \( m \) is its mass, and \( a \) is the resulting acceleration.

In this experiment, we aim to investigate the relationship between net force and acceleration using a cart system. The experimental setup consists of a cart with a constant weight placed on top, connected to a pulley system that allows for varying weights to be attached. By manipulating the mass of the weights pulling the cart and measuring the time taken for the cart to travel a predetermined distance, we can calculate the acceleration of the cart for different net forces applied.

In the context of our setup, the acceleration of the cart can be expressed by the equation:

\begin{equation}
    a = \frac{m_2 g}{m_1 + m_2}
\end{equation}

where \( m_2 \) is the hanging mass, \( g \) is the acceleration due to gravity, \( m_1 \) is the mass of the cart.

We hypothesize that:

Alternative Hypothesis (\( H_1 \)): 
\[
H_1: a \propto F
\]
As the net force acting on the cart increases due to the addition of weights, the acceleration of the cart will also increase, demonstrating a direct proportionality between net force and acceleration as stated in Newton's Second Law [1].

Null Hypothesis (\( H_0 \)): 
\[
H_0: a = k
\]
There is no significant relationship between the net force acting on the cart and the acceleration of the cart; the acceleration remains constant regardless of changes in net force.

This experiment not only aims to validate this fundamental law but also provides insights into the practical applications of these principles in real-world scenarios, such as in vehicles and various mechanical systems.

\textcolor{black}{Page limit: 2}
