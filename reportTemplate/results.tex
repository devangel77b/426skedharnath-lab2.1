\section{Results}

For each trial, the acceleration \( a \) was calculated using the derived equation based on the distance and time, without referencing masses.

\subsection{Trial Data Summary}

\begin{table}[h!]
    \centering
    \begin{tabular}{|c|c|c|c|}
        \hline
        Trial & Mass of Cart (\(m_1\)) & Weight Pulling Cart (\(m_2\)) & Time (\(t\)) \\
        \hline
        1 & 780 g & 20 g &  2.35 s\\
        2 & 780 g & 50 g & 1.45 s\\
        3 & 780 g & 100 g & 1.20 s\\
        4 & 780 g & 200 g & 0.79 s\\
        5 & 780 g & 500 g &  0.62 s\\
        \hline
    \end{tabular}
    \caption{Summary of trial data.}
\end{table}
  
\subsection{Calculations and Results}

The acceleration for each trial was determined using the equation:

\[
a = \frac{2d}{t^2}
\]

where \( d = 0.65   \, \text{m} \) is the acceleration due to gravity. The calculated accelerations for each trial are as follows:

For Trial 1: with \( m_2 = 0.020 \, \text{kg} \), we calculate
\[
a_1 = \frac{2 \times 0.65}{2.35^2} = \frac{1.3}{5.5225} \approx 0.235 \, \text{m/s}^2
\]

For Trial 2: with \( m_2 = 0.050 \, \text{kg} \), we calculate
\[
a_2 = \frac{2 \times 0.65}{1.45^2} = \frac{1.3}{2.1025} \approx 0.617 \, \text{m/s}^2
\]

For Trial 3: with \( m_2 = 0.100 \, \text{kg} \), we calculate
\[
a_3 = \frac{2 \times 0.65}{1.20^2} = \frac{1.3}{1.44} \approx  0.902 \, \text{m/s}^2
\]

For Trial 4: with \( m_2 = 0.200 \, \text{kg} \), we calculate
\[
a_4 = \frac{2 \times 0.65}{0.79^2} = \frac{1.3}{0.6241} \approx 2.083 \, \text{m/s}^2
\]

For Trial 5: with \( m_2 = 0.500 \, \text{kg} \), we calculate
\[
a_5 = \frac{2 \times 0.65}{0.62^2} = \frac{1.3}{0.3844} \approx 3.384 \, \text{m/s}^2
\]

\subsection{Summary of Results}

The calculated accelerations for each trial are summarized below:

\begin{itemize}
    \item Trial 1: \( a_1 \approx 0.235 \, \text{m/s}^2 \)
    \item Trial 2: \( a_2 \approx 0.617 \, \text{m/s}^2 \)
    \item Trial 3: \( a_3 \approx 0.902 \, \text{m/s}^2 \)
    \item Trial 4: \( a_4 \approx 2.083 \, \text{m/s}^2 \)
    \item Trial 5: \( a_5 \approx 3.384 \, \text{m/s}^2 \)
\end{itemize}

These values represent the observed accelerations for each configuration of mass, calculated based on the gravitational force and Newton's Second Law [1].

Predicted accelerations:
Using the formula \( a \approx m_2/(m_1+m_2) g \) we calculated the predicted accelerations below:
\begin{itemize}
    \item Trial 1: \( a_1 \approx 0.245 \, \text{m/s}^2 \)
    \item Trial 2: \( a_2 \approx 0.588 \, \text{m/s}^2 \)
    \item Trial 3: \( a_3 \approx 1.117 \, \text{m/s}^2 \)
    \item Trial 4: \( a_4 \approx 2.000 \, \text{m/s}^2 \)
    \item Trial 5: \( a_5 \approx 3.832 \, \text{m/s}^2 \)
\end{itemize}




\begin{figure}
    \centering
    \includegraphics[width=1.0\linewidth]{graph_final.png}
    \caption{Graph of mass ratio vs acceleration}
    \label{fig:graph}
\end{figure}
