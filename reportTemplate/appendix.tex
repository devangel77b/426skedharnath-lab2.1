\section{Appendix}

\subsection*{Derivations of Formulas}

\subsection{Acceleration of an Atwood Machine}


To derive the formula for the acceleration \(a\) of the system, we analyze the forces acting on both masses \(m_1\) and \(m_2\). The force analysis for each mass is as follows:

For mass \( m_1 \), the only horizontal force acting on it is the tension \( T \) in the string, leading to the equation:
\[
T = m_1a \quad \text{(Equation 1)}
\]
For mass \( m_2 \), the forces acting on it are the downward force due to gravity (\( m_2g \)) and the upward tension \( T \) in the string. The net force on mass \( m_2 \) is given by:
\[
m_2g - T = m_2a \quad \text{(Equation 2)}
\]

Next, substitute Equation 1 into Equation 2 to eliminate \( T \):
\[
m_2g - m_1a = m_2a
\]
Rearranging this equation:
\[
m_1a + m_2a = m_2g
\]
Factor out \( a \) from the left-hand side:
\[
a(m_1 + m_2) = m_2g
\]
Solving for \( a \):
\[
a = \frac{m_2g}{m_1 + m_2} \quad \text{(Equation 3)}
\]

Thus, the acceleration \( a \) of the system is given by the formula:
\[
a = \frac{m_2g}{m_1 + m_2}
\]
This equation shows that the acceleration depends on the masses \( m_1 \) and \( m_2 \), and the gravitational acceleration \( g \).

\subsection{Kinematic Equation for Constant Acceleration.}

The kinematic equation for an object undergoing uniformly accelerated motion is given by:
\[
d = v_0 t + \frac{1}{2} a t^2,
\]
where:
\begin{itemize}
    \item \( d \) is the displacement,
    \item \( v_0 \) is the initial velocity,
    \item \( a \) is the acceleration,
    \item \( t \) is the time.
\end{itemize}

If the object starts from rest, the initial velocity is \( v_0 = 0 \). Substituting \( v_0 = 0 \) into the equation simplifies it to:
\[
d = \frac{1}{2} a t^2.
\]

Rearranging this equation to solve for \( a \), we get:
\[
a = \frac{2d}{t^2}.
\]

Thus, the acceleration \( a \) is expressed in terms of the distance \( d \) traveled and the time \( t \) taken.